\documentclass{beamer}
\usepackage{../tut-slides}
\usepackage{../mathoperatorsAuD}

\usepackage{amsmath,amssymb}
\usepackage{stmaryrd}
\usepackage{enumerate}
%\usepackage[inline]{enumitem} 		%customize label
%\newcommand{\labelitemi}{\raisebox{1pt}{\scalebox{.9}{$\blacktriangleright$}}}
%\newcommand{\labelitemii}{$\vartriangleright$}
%\newcommand{\labelitemiii}{--}
\setbeamertemplate{itemize item}{\raisebox{1pt}{\scalebox{.9}{$\blacktriangleright$}}}
\setbeamertemplate{itemize subitem}{$\vartriangleright$}

\usepackage{booktabs}
\usepackage{tabularx}
\usepackage{tabu}
\newcommand*\head{\rowfont{\bfseries}}
\newcommand*{\tw}{\rowfont{\ttfamily}}
\renewcommand{\tabularxcolumn}[1]{>{\hspace{0pt}}m{#1}}

\usepackage{cancel}

\usepackage{empheq}
\newcommand*\widefbox[1]{\fbox{\hspace{2em} #1 \hspace{2em}}}

\usepackage{tcolorbox}
\newtcolorbox{mymathbox}[1][]{colback=white, sharp corners, #1}

%%%% EBNF-Terme %%%%
\newcommand{\wdh}[1]{\hat{\{} \ #1 \ \hat{\}}}
\newcommand{\opt}[2]{\hat{(} \ #1 \ \hat{|} \ #2 \ \hat{)}}
\newcommand{\byp}[1]{\hat{[} \ #1 \ \hat{]}}
\newcommand{\rdb}[1]{\hat{(} \ #1 \ \hat{)}}

\newcommand{\sem}[1]{\left\llbracket #1 \right\rrbracket}

\usepackage{xcolor}
\usepackage{listings}
\lstset{numbers=left, 
	numberstyle=\tiny, 
	breaklines=true,
	backgroundcolor=\color{cdgray!20},
	numbersep=5pt,
	language=C,
	tabsize=2,
	basicstyle=\footnotesize} 


\begin{document}	
	\title{Algorithmen und Datenstrukturen}
	\subtitle{Übung 5: Programmieren mit $C$}
	\author{Eric Kunze}
	\email{eric.kunze@mailbox.tu-dresden.de}
	\city{TU Dresden}
%	\institute{Lehrstuhl für Grundlagen der Programmierung}
	\titlegraphic{\includegraphics[width=2cm]{../TUD-white.pdf}}
	\date{21.11.2019}

	\maketitle


%%%%%%%%%%%%%%%%%%%%%%%%%%%%%%%%%%%%%%%%%%%%%%%%%%%%%%%%%%%%%%%%%%%%%%%%%%%%%

\begin{frame}[fragile] \frametitle{C}
	\begin{itemize}
		\item Input / Output: \lstinline{# include <stdio.h>}
		\item Variablentypen: z.B. \texttt{int}, \texttt{float}, \texttt{char}
		\item arithmetische Operatoren: \texttt{+},\texttt{-},\texttt{*},\texttt{/},\texttt{\%}
		\item Vergleichsoperatoren: \texttt{==}, \texttt{<}, \texttt{<=}, \texttt{>}, \texttt{>=}
		\item Logikoperatoren: \texttt{!}, \texttt{\&\&}, \texttt{||}
		\item Arrays: \lstinline[basicstyle=\normalsize]|int feld[7]| (Indizierung beginnend bei 0)
	\end{itemize}
\end{frame}

%%%%%%%%%%%%%%%%%%%%%%%%%%%%%%%%%%%%%%%%%%%%%%%%%%%%%%%%%%%%%%%%%%%%%%%%%%%%%%%%%%%

\begin{frame}[fragile] \frametitle{Bedingungen}
	\begin{itemize}
		\item If-Else-Statement: bedingte Ausführung eines Statements
\begin{lstlisting}
if ( BoolExp ) {
	Statement ;
} else  {
	Statement ;
}
\end{lstlisting}
		\item Switch-Statement: Fallunterscheidung mit mehr als zwei Fällen
\begin{lstlisting}
switch ( Exp ) {
	case 0: StatementSeq ; break ;
	case 1: StatementSeq ; break ;
	default: StatementSeq ;
}
\end{lstlisting}
		
	\end{itemize}
\end{frame}

%%%%%%%%%%%%%%%%%%%%%%%%%%%%%%%%%%%%%%%%%%%%%%%%%%%%%%%%%%%%%%%%%%%%%%%%%%%%%%%%%%%

\begin{frame}[fragile] \frametitle{Schleifen}
	\begin{itemize}
		\item While-Statement: wiederholte Ausführung eines Statements (Schleifenrumpf)
\begin{lstlisting}
while ( BoolExp ) {
	Statement ;
}
\end{lstlisting}
		\item Do-While-Statement: vergleichbar mit While-Statement, aber Schleifenbedingung wird nach Rumpf geprüft
\begin{lstlisting}
do {
	Statement ;
} while ( BoolExp )
\end{lstlisting}

	\end{itemize}
\end{frame}

%%%%%%%%%%%%%%%%%%%%%%%%%%%%%%%%%%%%%%%%%%%%%%%%%%%%%%%%%%%%%%%%%%%%%%%%%%%%%%%%%%%

\begin{frame}[fragile] \frametitle{Schleifen}
	\begin{itemize}
		\item For-Statement: vor der Schleife steht Anzahl der Schleifendurchläufe fest
\begin{lstlisting}[language=C]
for ( Assignment ; BoolExp ; Assignment ) {
	Statement ;
}
\end{lstlisting}		
	\end{itemize}
\end{frame}

%%%%%%%%%%%%%%%%%%%%%%%%%%%%%%%%%%%%%%%%%%%%%%%%%%%%%%%%%%%%%%%%%%%%%%%%%%%%%%%%%%%

\end{document}